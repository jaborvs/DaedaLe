\chapter*{Abstract}
% Here goes your abstract. Be concise, introduce context, problem, known approaches, your solution, your findings.

The digital game industry is a thriving multi-billion dollar industry. Creating interesting and interactive games is a complex process with two major parts: game design and game development. Game design is the art of applying design to create games on a conceptual level. Game development is the process of bringing the game design to life. Part of the game design process is the creation of game mechanics, rules that define how the player interacts with the game. Playtesting is the process of evaluating the impact of these rules on the player experience, with the goal being a net positive impact. However, playtesting has a significant resource and time cost associated with it, as such game designers must sometimes make decisions when evolving their game without the necessary knowledge of the impact on the player experience. 

% Game design is a complex process with the aim of creating interactive and interesting games, game development is that process implemented into reality. 

% Part of the game design process is the creation of game mechanics. Game mechanics define how a player can interact with the game and play a key role in the process.

% As such, playtesting of game mechanics is essential but currently, only two approaches exist: manual or AI-based. Both are very time-consuming and therefore cannot be run every time changes are made to the game mechanics. As a result, game designers do not always fully understand the effect of changes they make to the game mechanics and must therefore take potentially risky decisions when evolving their game.

% As a result game designers must take risks when evolving their games.

% We propose an alternative solution to this issue from a meta-programming perspective. We leverage meta-programming techniques to provide game designers with feedback on the quality of their games.

We approach the study of this problem from a meta-programming perspective. We aim to empower game designers with tools and techniques that give feedback about the quality of the games. In particular, we study how dynamic analyses can provide live feedback about a game's rules. We focus our efforts on a concrete problem by studying PuzzleScript and evaluating our approach on a set of published games written using that engine.


% We propose a system of dynamic analysis solutions that focus on giving rapid feedback on the quality of game mechanics and code.

% We evaluate this approach on published games written using the PuzzleScript engine.

% PuzzleScript is a browser-based game engine where source code is closely tied to game mechanics.

% To aid our initial goal and future research in PuzzleScript, formalize the design of PuzzleScript and implement a redesign of of the technical implementation using Rascal, a meta-programming inclined language. We then extend our implementation by adding our system of solution which we can then test on existing

We formalize the design of PuzzleScript and implement a redesign of the technical implementation using Rascal, a language workbench designed to facilitate meta-programming. This more extensible and maintainable prototype implementation of PuzzleScript aids us in our initial goal and in future PuzzleScript research. We then extend our implementation with our system of game mechanics analysis and test games for game mechanic errors. Finally, we validate our approach on a set of real-world published games and modify games to test for gameplay decay, the fall in gameplay quality as a result of evolution in-game mechanics.